\documentclass{amsart}
\usepackage{attapl}
\usepackage{mathtools}

\title{Advanced Topics in Types and Programming Languages\\
    Ch 10: The Essence of ML Type Inference\\Sec 10.1: What is ML?}
%\title{Advanced Topics in Types and Programming Languages\\
%    Editor: Benjamin C. Pierce\\Ch 10: The Essence of ML Type Inference\\
%    By Fran\c{c}ois Pottier and Dider R\'emy\\Sec 10.1: What is ML?}
\date{Saturday, 2016-11-12}

\begin{document}
\maketitle

\def\dto{\overset{\delta}{\to}}

\newlength{\colwid}
\setlength{\colwid}{0.5\textwidth}

\newcommand{\p}{$^\prime$}

% Figure 10-2: Semantics of ML-the-calculus
\newlength{\oldjot}
\setlength{\oldjot}{\jot}
\begin{figure}[h]
  \begin{tabular}{p{\colwid}|p{\colwid}}
    {\begin{minipage}{\colwid}
      \setlength{\jot}{1em}
      \begin{align*}
        (\lambda \mono{z}.\mono{t}) \mono{ v} \to [\mono{z} \mapsto
            \mono{v}]\mono{t}& & &\textup{(R-Beta)} \\
        \mono{let z} = \mono{v in t} \to [\mono{z} \mapsto
            \mono{v}]\mono{t}& & &\textup{(R-Let)} \\
        \frac{\mono{t}/\mu \dto \mono{t}'/\mu'}{\mono{t}/\mu \to
            \mono{t}'/\mu'}& & &\textup{(R-Delta)}
      \end{align*}
    \end{minipage}}
    &
    {\begin{minipage}{\colwid}
      \begin{align*}
      \frac{
          \begin{gathered}
            \mono{t}/\mu \to \mono{t}'/\mu' \\
            dom(\mu'') \disjoint dom(\mu') \\
            range(\mu'') \disjoint dom(\mu'\setminus\mu)
          \end{gathered}
        }{\mono{t}/\mu\mu'' \to \mono{t}'/\mu'\mu''}&
        & &\textup{(R-Extend)} \\[1em]
      \frac{\mono{t}/\mu \to \mono{t}'/\mu'}
          {\mathcal{E}[\mono{t}]/\mu \to
          \mathcal{E}[\mono{t}']/\mu'}& & &\textup{(R-Context)}
    \end{align*} \end{minipage}}
  \end{tabular}
  \caption{[10-2] Semantics of ML-the-calculus}
\end{figure}

% Exercise 10.1.6
\begin{problem}[10.1.6]
  Let \mono{true} and \mono{false} be nullary constructors. Let \mono{if} be a
  ternary destructor. Extend the semantics with
  \begin{align*}
    \mono{if true v}_1 \mono{ v}_2 &\overset{\delta}{\to} \mono{v}_1 &
        &\textup{(R-True)} \\
    \mono{if false v}_1 \mono{ v}_2 &\overset{\delta}{\to} \mono{v}_2 &
        &\textup{(R-False)}
  \end{align*}
  Let us use the syntactic sugar \(\mono{if t}_0 \mono{ then t}_1
  \mono{ else t}_2\) for the triple application of \(\mono{if t}_0 \mono{ t}_1
  \mono{ t}_2\). Explain why these definitions do not quite provide the expected
  behavior. Without modifying the semantics of \mono{if}, suggest a new
  definition of the syntactic sugar \(\mono{if t}_0 \mono{ then t}_1
  \mono{ else t}_2\) that corrects the problem.
\end{problem}

\begin{solution}
  Desugaring to \(\mono{if t}_0 \mono{ t}_1 \mono{ t}_2\) will cause
  both \(\mono{t}_1\) and \(\mono{t}_2\) to be fully evaluated regardless of the
  value of \(\mono{t}_0\) because the \(\mono{if}\) destructor cannot be reduced
  until all of its arguments already have been. This is not the expected
  behavior if \(\mono{t}_1\) or \(\mono{t}_2\) has side effects.

  One way to resolve the issue is to desugar instead to \((\mono{if t}_0
  \mono{ } (\lambda \mono{\_.t}_1) \mono{ } (\lambda \mono{\_.t}_2))
  \mono{ p}\) where \(\mono{p}\) is an arbitrary pure term (without side
  effects).  Wrapping \(\mono{t}_1\) and \(\mono{t}_2\) in
  \(\lambda\)-abstractions (which are values) defers the evaluation
  of \(\mono{t}_1\) (\(\mono{t}_2\)) until the corresponding
  \(\lambda\)-abstraction is applied. Since only the right one is ever applied,
  only the right term is ever evaluated.
\end{solution}

% Exercise 10.1.8
\begin{problem}[10.1.8]
  Explain how to encode \(\mono{true}\), \(\mono{false}\), and the \(\mono{if}\)
  construct in terms of sums. Check that the behavior of \(\textup{(R-True)}\)
  and \(\textup{(R-False)}\) is properly emulated.
\end{problem}

\begin{solution}
  This amounts to a minor rearrangement of the solution to 10.1.6. Let
  \(\mono{p}\) be an arbitrary pure term as before. Define \(\mono{true}\)
  (\(\mono{false}\)) as syntactic sugar for \(\mono{inj}_1 \mono{ p}\)
  (\(\mono{inj}_2 \mono{ p}\)), and \(\mono{if t}_0 \mono{ t}_1 \mono{ t}_2\) as
  syntactic sugar for \(\mono{case t}_0 \mono{ } (\lambda \mono{\_.t}_1)
  \mono{ } (\lambda \mono{\_.t}_2)\). We verify the reductions by calculation:
  \begin{align*}
    \mono{if true then t}_1 \mono{ else t}_2 &\mapsto \mono{case }
        (\mono{inj}_1 \mono{ p}) \mono{ } (\lambda \mono{\_.t}_1) \mono{ }
        (\lambda \mono{\_.t}_2) \\
    &\to (\lambda \mono{\_.t}_1) \mono{ p} \\
    &\to \mono{v}_1 \\
    \mono{if false then t}_1 \mono{ else t}_2 &\mapsto \mono{case }
        (\mono{inj}_2 \mono{ p}) \mono{ } (\lambda \mono{\_.t}_1) \mono{ }
        (\lambda \mono{\_.t}_2) \\
    &\to (\lambda \mono{\_.t}_2) \mono{ p} \\
    &\to \mono{v}_2
  \end{align*}
  \qed
\end{solution}

% Exercise 10.1.11
\begin{problem}[10.1.11]
  Assuming the availability of Booleans and conditionals, integer literals,
  subtraction, multiplication, integer comparison, and a fixpoint combinator,
  most of which were defined in previous examples, define a function that
  computes the factorial of its integer argument, and applit to \(\hat{3}\).
  Determine, step by step, how this expression reduces to a value.
\end{problem}

\begin{solution}
  We may define such a function as
  \begin{align*}
    \mono{fact} := \mono{fix } \lambda \mono{f.} \lambda \mono{n.} \!
        \left(\mono{case } (\hat{\mono{<}} \mono{ n } \hat{\mono{2}}) \mono{ }
        \lambda \mono{\_.} \! (\hat{\mono{1}}) \mono{ }
        \lambda \mono{\_.} \! (\hat{\mono{*}} \mono{ n } (\mono{f }
          (\hat{\mono{-}} \mono{ n } \hat{\mono{1}})))\right)
  \end{align*}
  Evaluating \(\mono{fact } \hat{\mono{3}}\) we find that
  \begin{align*}
    \mono{fact } \hat{\mono{3}} \:\: \mapsto \:\:& \mono{fix }
        \lambda \mono{f.} \lambda \mono{n.} \!
          \left(\mono{case } (\hat{\mono{<}} \mono{ n } \hat{\mono{2}}) \mono{ }
          \lambda \mono{\_.} \! (\hat{\mono{1}}) \mono{ }
          \lambda \mono{\_.} \! (\hat{\mono{*}} \mono{ n } (\mono{f }
            (\hat{\mono{-}} \mono{ n } \hat{\mono{1}})))\right) \mono{ }
        \hat{\mono{3}}
        \\[0.5em]
    \to \:\: &\lambda \mono{f.} \lambda \mono{n.} \!
        \left(\mono{case } (\hat{\mono{<}} \mono{ n } \hat{\mono{2}}) \mono{ }
          \lambda \mono{\_.} \! (\hat{\mono{1}}) \mono{ }
          \lambda \mono{\_.} \! (\hat{\mono{*}} \mono{ n } (\mono{f }
            (\hat{\mono{-}} \mono{ n } \hat{\mono{1}})))\right) \\
    &\mono{ } \left(\mono{fix } \lambda \mono{f.} \lambda \mono{n.} \!
        \left(\mono{case } (\hat{\mono{<}} \mono{ n } \hat{\mono{2}}) \mono{ }
          \lambda \mono{\_.} \! (\hat{\mono{1}}) \mono{ }
          \lambda \mono{\_.} \! (\hat{\mono{*}} \mono{ n } (\mono{f }
            (\hat{\mono{-}} \mono{ n } \hat{\mono{1}})))\right)\right) \mono{ }
        \hat{\mono{3}}
        \\[0.5em]
    \to \:\: &\lambda \mono{n.} \!
        \left(\mono{case } (\hat{\mono{<}} \mono{ n } \hat{\mono{2}}) \mono{ }
          \lambda \mono{\_.} \! (\hat{\mono{1}}) \mono{ }
          \lambda \mono{\_.} \! \left(\hat{\mono{*}} \mono{ n} \right. \right.
          \\
    \:\: &\qquad \left. \left. \left(\mono{fix }
        \lambda \mono{f.} \lambda \mono{n.} \!
          \left(\mono{case } (\hat{\mono{<}} \mono{ n } \hat{\mono{2}}) \mono{ }
            \lambda \mono{\_.} \! (\hat{\mono{1}}) \mono{ }
            \lambda \mono{\_.} \! (\hat{\mono{*}} \mono{ n } (\mono{f }
              (\hat{\mono{-}} \mono{ n } \hat{\mono{1}})))\right) \mono{ }
            (\hat{\mono{-}} \mono{ n } \hat{\mono{1}})\right)\right)\right)
        \mono{ } \hat{\mono{3}}
        \\[0.5em]
    \to \:\: & \mono{case }
        (\hat{\mono{<}}\mono{ }\hat{\mono{3}}\mono{ }\hat{\mono{2}}) \mono{ }
          \lambda \mono{\_.} \! (\hat{\mono{1}}) \mono{ }
          \lambda \mono{\_.} \! \left(\hat{\mono{*}} \mono{ }\hat{\mono{3}}
          \right. \\
    \:\: &\qquad \left. \left(\mono{fix }
        \lambda \mono{f.} \lambda \mono{n.} \!
          \left(\mono{case } (\hat{\mono{<}} \mono{ n } \hat{\mono{2}}) \mono{ }
            \lambda \mono{\_.} \! (\hat{\mono{1}}) \mono{ }
            \lambda \mono{\_.} \! (\hat{\mono{*}} \mono{ n } (\mono{f }
              (\hat{\mono{-}} \mono{ n } \hat{\mono{1}})))\right) \mono{ }
            (\hat{\mono{-}}\mono{ }\hat{\mono{3}}\mono{ }\hat{\mono{1}})\right)
          \right)
  \end{align*}
  \begin{align*}
    \to \:\: & \mono{case false }
          \lambda \mono{\_.} \! (\hat{\mono{1}}) \mono{ }
          \lambda \mono{\_.} \! \left(\hat{\mono{*}} \mono{ }\hat{\mono{3}}
          \right. \\
    \:\: &\qquad \left. \left(\mono{fix }
        \lambda \mono{f.} \lambda \mono{n.} \!
          \left(\mono{case } (\hat{\mono{<}} \mono{ n } \hat{\mono{2}}) \mono{ }
            \lambda \mono{\_.} \! (\hat{\mono{1}}) \mono{ }
            \lambda \mono{\_.} \! (\hat{\mono{*}} \mono{ n } (\mono{f }
              (\hat{\mono{-}} \mono{ n } \hat{\mono{1}})))\right) \mono{ }
            (\hat{\mono{-}}\mono{ }\hat{\mono{3}}\mono{ }\hat{\mono{1}})\right)
          \right)
        \\[0.5em]
    \to \:\: & \lambda \mono{\_.} \! \left(\hat{\mono{*}} \mono{ }\hat{\mono{3}}
        \mono{ } \! \left(\mono{fix } \lambda \mono{f.} \lambda \mono{n.} \!
          \left(\mono{case } (\hat{\mono{<}} \mono{ n } \hat{\mono{2}}) \mono{ }
            \lambda \mono{\_.} \! (\hat{\mono{1}}) \mono{ }
            \lambda \mono{\_.} \! (\hat{\mono{*}} \mono{ n } (\mono{f }
              (\hat{\mono{-}} \mono{ n } \hat{\mono{1}})))\right) \mono{ }
            (\hat{\mono{-}}\mono{ }\hat{\mono{3}}\mono{ }\hat{\mono{1}})\right)
          \right) \mono{ p}
        \\[0.5em]
    \to \:\: & \hat{\mono{*}} \mono{ }\hat{\mono{3}}
        \mono{ } \! \left(\mono{fix } \lambda \mono{f.} \lambda \mono{n.} \!
          \left(\mono{case } (\hat{\mono{<}} \mono{ n } \hat{\mono{2}}) \mono{ }
            \lambda \mono{\_.} \! (\hat{\mono{1}}) \mono{ }
            \lambda \mono{\_.} \! (\hat{\mono{*}} \mono{ n } (\mono{f }
              (\hat{\mono{-}} \mono{ n } \hat{\mono{1}})))\right) \mono{ }
            (\hat{\mono{-}}\mono{ }\hat{\mono{3}}\mono{ }\hat{\mono{1}})\right)
        \\[0.5em]
    \to \:\: & \hat{\mono{*}} \mono{ }\hat{\mono{3}}
        \mono{ } \! \left(\lambda \mono{f.} \lambda \mono{n.} \!
          \left(\mono{case } (\hat{\mono{<}} \mono{ n } \hat{\mono{2}}) \mono{ }
            \lambda \mono{\_.} \! (\hat{\mono{1}}) \mono{ }
            \lambda \mono{\_.} \! (\hat{\mono{*}} \mono{ n } (\mono{f }
              (\hat{\mono{-}} \mono{ n } \hat{\mono{1}})))\right)
        \right. \\
    \:\: &\qquad \left. \left(\mono{fix } \lambda \mono{f.} \lambda \mono{n.} \!
          \left(\mono{case } (\hat{\mono{<}} \mono{ n } \hat{\mono{2}}) \mono{ }
            \lambda \mono{\_.} \! (\hat{\mono{1}}) \mono{ }
            \lambda \mono{\_.} \! (\hat{\mono{*}} \mono{ n } (\mono{f }
              (\hat{\mono{-}} \mono{ n } \hat{\mono{1}})))\right)\right)
            \mono{ }
            (\hat{\mono{-}}\mono{ }\hat{\mono{3}}\mono{ }\hat{\mono{1}})\right)
  \end{align*}
  \begin{align*}
    \to \:\: & \hat{\mono{*}} \mono{ }\hat{\mono{3}}
        \mono{ } \! \left(\lambda \mono{f.} \lambda \mono{n.} \!
          \left(\mono{case } (\hat{\mono{<}} \mono{ n } \hat{\mono{2}}) \mono{ }
            \lambda \mono{\_.} \! (\hat{\mono{1}}) \mono{ }
            \lambda \mono{\_.} \! (\hat{\mono{*}} \mono{ n } (\mono{f }
              (\hat{\mono{-}} \mono{ n } \hat{\mono{1}})))\right)
        \right. \\
    \:\: &\qquad \left. \left(\mono{fix } \lambda \mono{f.} \lambda \mono{n.} \!
          \left(\mono{case } (\hat{\mono{<}} \mono{ n } \hat{\mono{2}}) \mono{ }
            \lambda \mono{\_.} \! (\hat{\mono{1}}) \mono{ }
            \lambda \mono{\_.} \! (\hat{\mono{*}} \mono{ n } (\mono{f }
              (\hat{\mono{-}} \mono{ n } \hat{\mono{1}})))\right)\right)
          \mono{ } \hat{\mono{2}}\right)
        \\[0.5em]
    \to \:\: & \hat{\mono{*}} \mono{ }\hat{\mono{3}}
        \mono{ } \! \left(\lambda \mono{n.} \!
          \left(\mono{case } (\hat{\mono{<}} \mono{ n } \hat{\mono{2}}) \mono{ }
            \lambda \mono{\_.} \! (\hat{\mono{1}}) \mono{ }
            \lambda \mono{\_.} \! (\hat{\mono{*}} \mono{ n } \right. \right. \\
    \:\:&\qquad\quad\left.\left.(\mono{fix }\lambda\mono{f'.}\lambda\mono{n'.}\!
          \left(\mono{case } (\hat{\mono{<}} \mono{ n' } \hat{\mono{2}})
            \mono{ } \lambda \mono{\_.} \! (\hat{\mono{1}}) \mono{ }
            \lambda \mono{\_.} \! (\hat{\mono{*}} \mono{ n' } (\mono{f' }
              (\hat{\mono{-}} \mono{ n' } \hat{\mono{1}})))\right)
        \mono{ } (\hat{\mono{-}} \mono{ n } \hat{\mono{1}})))\right) \mono{ }
        \hat{\mono{2}}\right)
        \\[0.5em]
    \to \:\: & \hat{\mono{*}} \mono{ }\hat{\mono{3}}
        \mono{ } \! \left(\mono{case }
          (\hat{\mono{<}}\mono{ }\hat{\mono{2}}\mono{ }\hat{\mono{2}}) \mono{ }
          \lambda \mono{\_.} \! (\hat{\mono{1}}) \mono{ }
          \lambda \mono{\_.} \! (\hat{\mono{*}} \mono{ }\hat{\mono{2}}\mono{ }
          \right. \\
    \:\: &\qquad\quad \left. (\mono{fix }\lambda\mono{f'.}\lambda\mono{n'.}\!
          \left(\mono{case } (\hat{\mono{<}} \mono{ n' } \hat{\mono{2}})
            \mono{ } \lambda \mono{\_.} \! (\hat{\mono{1}}) \mono{ }
            \lambda \mono{\_.} \! (\hat{\mono{*}} \mono{ n' } (\mono{f' }
              (\hat{\mono{-}} \mono{ n' } \hat{\mono{1}})))\right) \mono{ }
        (\hat{\mono{-}}\mono{ }\hat{\mono{2}}\mono{ }\hat{\mono{1}})))\right)
        \\[0.5em]
    \to \:\: & \hat{\mono{*}} \mono{ }\hat{\mono{3}}
        \mono{ } \! \left(\mono{case false }
          \lambda \mono{\_.} \! (\hat{\mono{1}}) \mono{ }
          \lambda \mono{\_.} \! (\hat{\mono{*}} \mono{ }\hat{\mono{2}}\mono{ }
          \right. \\
    \:\: &\qquad\quad \left. (\mono{fix }\lambda\mono{f'.}\lambda\mono{n'.}\!
          \left(\mono{case } (\hat{\mono{<}} \mono{ n' } \hat{\mono{2}})
            \mono{ } \lambda \mono{\_.} \! (\hat{\mono{1}}) \mono{ }
            \lambda \mono{\_.} \! (\hat{\mono{*}} \mono{ n' } (\mono{f' }
              (\hat{\mono{-}} \mono{ n' } \hat{\mono{1}})))\right) \mono{ }
        (\hat{\mono{-}}\mono{ }\hat{\mono{2}}\mono{ }\hat{\mono{1}})))\right)
  \end{align*}
  \begin{align*}
    \to \:\: & \hat{\mono{*}} \mono{ }\hat{\mono{3}}
        \mono{ } \! \left(\lambda \mono{\_.} \!
          (\hat{\mono{*}} \mono{ } \hat{\mono{2}} \mono{ } \right. \\
    \:\: &\qquad\quad \left. (\mono{fix }\lambda\mono{f'.}\lambda\mono{n'.}\!
          \left(\mono{case } (\hat{\mono{<}} \mono{ n' } \hat{\mono{2}})
            \mono{ } \lambda \mono{\_.} \! (\hat{\mono{1}}) \mono{ }
            \lambda \mono{\_.} \! (\hat{\mono{*}} \mono{ n' } (\mono{f' }
              (\hat{\mono{-}} \mono{ n' } \hat{\mono{1}})))\right) \mono{ }
        (\hat{\mono{-}}\mono{ }\hat{\mono{2}}\mono{ }\hat{\mono{1}})) \mono{ p})
        \right)
        \\[0.5em]
    \to \:\: & \hat{\mono{*}} \mono{ } \hat{\mono{3}} \mono{ } \!
        \left(\hat{\mono{*}} \mono{ } \hat{\mono{2}} \mono{ }
            (\mono{fix } \lambda \mono{f'.} \lambda\mono{n'.} \!
              \left(\mono{case } (\hat{\mono{<}} \mono{ n' } \hat{\mono{2}})
                \mono{ }
                \lambda \mono{\_.} \! (\hat{\mono{1}}) \mono{ }
                \lambda \mono{\_.} \! (\hat{\mono{*}} \mono{ n' } (\mono{f' }
                  (\hat{\mono{-}} \mono{ n' } \hat{\mono{1}})))\right) \mono{ }
              (\hat{\mono{-}}\mono{ }\hat{\mono{2}}\mono{ }\hat{\mono{1}}))
        \right)
        \\[0.5em]
    \to \:\: & \hat{\mono{*}} \mono{ } \hat{\mono{3}} \mono{ } \!
        \left(\hat{\mono{*}} \mono{ } \hat{\mono{2}} \mono{ }
            (\mono{fix } \lambda \mono{f'.} \lambda\mono{n'.} \!
              \left(\mono{case } (\hat{\mono{<}} \mono{ n' } \hat{\mono{2}})
                \mono{ }
                \lambda \mono{\_.} \! (\hat{\mono{1}}) \mono{ }
                \lambda \mono{\_.} \! (\hat{\mono{*}} \mono{ n' } (\mono{f' }
                  (\hat{\mono{-}} \mono{ n' } \hat{\mono{1}})))\right) \mono{ }
              \hat{\mono{1}})
        \right)
        \\[0.5em]
    \to \:\: & \hat{\mono{*}} \mono{ } \hat{\mono{3}} \mono{ } \!
        \left(\hat{\mono{*}} \mono{ } \hat{\mono{2}} \mono{ }
            \left(\lambda \mono{f'.} \lambda\mono{n'.} \!
              \left(\mono{case } (\hat{\mono{<}} \mono{ n' } \hat{\mono{2}})
                \mono{ }
                \lambda \mono{\_.} \! (\hat{\mono{1}}) \mono{ }
                \lambda \mono{\_.} \! (\hat{\mono{*}} \mono{ n' } (\mono{f' }
                  (\hat{\mono{-}} \mono{ n' } \hat{\mono{1}})))\right)
              \right. \right. \\
    \:\: &\qquad\quad\left.\left.
      \left(\mono{fix } \lambda \mono{f'.} \lambda\mono{n'.} \!
        \left(\mono{case } (\hat{\mono{<}} \mono{ n' } \hat{\mono{2}}) \mono{ }
          \lambda \mono{\_.} \! (\hat{\mono{1}}) \mono{ }
          \lambda \mono{\_.} \! (\hat{\mono{*}} \mono{ n' } (\mono{f' }
            (\hat{\mono{-}} \mono{ n' } \hat{\mono{1}})))\right)\right) \mono{ }
        \hat{\mono{1}}\right)\right)
  \end{align*}
  \begin{align*}
    \to \:\: & \hat{\mono{*}} \mono{ } \hat{\mono{3}} \mono{ } \!
        \left(\hat{\mono{*}} \mono{ } \hat{\mono{2}} \mono{ }
          \left(\mono{case }
            (\hat{\mono{<}} \mono{ } \hat{\mono{1}} \mono{ } \hat{\mono{2}})
              \mono{ }
            \lambda \mono{\_.} \! (\hat{\mono{1}}) \mono{ }
            \lambda \mono{\_.} \! (\hat{\mono{*}} \mono{ }\hat{\mono{1}}\mono{ }
            \right. \right. \\
    \:\: &\qquad \quad \left. \left.
        \left(\mono{fix }\lambda\mono{f''.}\lambda \mono{n''.}\!
          \left(\mono{case } (\hat{\mono{<}}\mono{ n'' }\hat{\mono{2}}) \mono{ }
            \lambda \mono{\_.} \! (\hat{\mono{1}}) \mono{ }
            \lambda \mono{\_.} \! (\hat{\mono{*}} \mono{ n'' } (\mono{f'' }
              (\hat{\mono{-}} \mono{ n'' } \hat{\mono{1}})))\right)\right)
        (\hat{\mono{-}} \mono{ }\hat{\mono{1}}\mono{ } \hat{\mono{1}}))
        \right)\right)
        \\[0.5em]
    \to \:\: & \hat{\mono{*}} \mono{ } \hat{\mono{3}} \mono{ } \!
        \left(\hat{\mono{*}} \mono{ } \hat{\mono{2}} \mono{ }
          \left(\lambda \mono{\_.} \! (\hat{\mono{1}}) \mono{ p}\right)\right)
        \\[0.5em]
    \to \:\: & \hat{\mono{*}} \mono{ } \hat{\mono{3}} \mono{ } \!
        \left(\hat{\mono{*}} \mono{ } \hat{\mono{2}} \mono{ } \hat{\mono{1}}
        \right)
        \\[0.5em]
    \to \:\: & \hat{\mono{*}} \mono{ } \hat{\mono{3}} \mono{ } \hat{\mono{2}}
        \\[0.5em]
    \to \:\: & \hat{\mono{6}}
  \end{align*}
  \qed
\end{solution}

\newpage

% Figure 10-3: Typing rules for DM
\begin{figure}[h]
  \begin{tabular}{p{\colwid}|p{\colwid}}
    {\begin{minipage}{\colwid}
      \setlength{\jot}{1em}
      \begin{align*}
        \frac{\Gamma(\mono{x}) = S}
             {\Gamma \entails \mono{x} \inhabits S}& &
          &\textup{(DM-Var)} \\
        \frac{\Gamma; \mono{z} \inhabits \mono{T} \entails
              \mono{t} \inhabits \mono{T}^\prime}
             {\Gamma \entails \lambda \mono{z.t} \inhabits \mono{T} \to
              \mono{T}^\prime}& &
          &\textup{(DM-Abs)} \\
        \frac{\Gamma \entails \mono{T}_1 \inhabits \mono{T} \to \mono{T}^\prime
              \quad \Gamma \entails \mono{t}_2 \inhabits \mono{T}}
             {\Gamma \entails \mono{t}_1\mono{ t}_2 \inhabits \mono{T}^\prime}&
          & &\textup{(DM-App)}
      \end{align*}
    \end{minipage}}
    &
    {\begin{minipage}{\colwid}
      \setlength{\jot}{1em}
      \begin{align*}
        \frac{\Gamma \entails \mono{t}_1 \inhabits S \quad \Gamma; \mono{z}
              \inhabits S \entails \mono{t}_2 \inhabits \mono{T}}
             {\Gamma \entails \mono{let z} = \mono{t}_1 \mono{ in t}_2 \inhabits
              \mono{T}}& &
          &\textup{(DM-Let)} \\
        \frac{\Gamma \entails \mono{t} \inhabits \mono{T} \quad
              \bar{\mono{X}} \disjoint ftv(\Gamma)}
             {\Gamma \entails \mono{t} \inhabits \forall\bar{\mono{X}}.\mono{T}}&
          & &\textup{(DM-Gen)} \\
        \frac{\Gamma \entails \mono{t} \inhabits \forall\bar{\mono{X}}.\mono{T}}
             {\Gamma \entails \mono{t} \inhabits [\vec{\mono{X}} \mapsto
              \vec{\mono{T}}]\mono{T}}& &
          &\textup{(DM-Inst)}
      \end{align*}
    \end{minipage}}
  \end{tabular}
  \caption{[10-3] Typing rules for DM}
\end{figure}

% Exercise 10.1.22
\begin{problem}[10.1.22]
  Build a type derivation for the expression \(\lambda \mono{z}_1 .
  \mono{let z}_2 = \mono{z}_1 \mono{ in z}_2.\)
\end{problem}

\begin{solution}
  \begin{prooftree}
    \AxiomC{\(
      \mono{z}_1 \inhabits \mono{T}
      \entails
      \mono{z}_1 \inhabits \mono{T}\)}
    \AxiomC{\(
      \mono{z}_1 \inhabits \mono{T}; \mono{z}_2 \inhabits \mono{T}
      \entails
      \mono{z}_2 \inhabits \mono{T}
    \)}
    \RightLabel{\small (DM-Let)}
    \BinaryInfC{\(
      \mono{z}_1 \inhabits \mono{T}
      \entails
      \mono{let z}_2 = \mono{z}_1
        \mono{ in z}_2 \inhabits \mono{T}
    \)}
    \RightLabel{\small (DM-Abs)}
    \UnaryInfC{\(
      \mono{z}_1 \inhabits \mono{T}
      \entails
      \lambda \mono{z}_1 .
        \mono{let z}_2 = \mono{z}_1 \mono{ in z}_2 \inhabits
        \mono{T} \to \mono{T}
    \)}
  \end{prooftree}
  \qed
\end{solution}

% Exercise 10.1.23
\begin{problem}
  Let \mono{int} be a nullary type constructor of signature \(\star\). Let
  \(\Gamma_0\) consist of the bindings \(\hat{+} \inhabits \mono{int} \to
  \mono{int} \to \mono{int}\) and \(\hat{k} \inhabits \mono{int}\), for every
  integer \(k\).  Can you find derivations of the following valid typing
  judgments? Which of these judgments are valid in the simply-typed
  \(\lambda\)-calculus, where \(\mono{let z} = \mono{t}_1 \mono{ in t}_2\) is
  syntactic sugar for \((\lambda \mono{z.t}_2) \mono{ t}_1\)?
  \begin{align*}
    \Gamma_0 \entails& \lambda \mono{z.z} \inhabits \mono{int} \to \mono{int} \\
    \Gamma_0 \entails& \lambda \mono{z.z} \inhabits \forall \mono{X.X} \to
      \mono{X} \\
    \Gamma_0 \entails& \mono{let f} = \lambda \mono{z.z } \hat{\mono{+}}
      \mono{ } \hat{\mono{1}} \mono{ in f } \hat{\mono{2}} \inhabits
      \mono{int} \\
    \Gamma_0 \entails& \mono{let f} = \lambda \mono{z.z} \mono{ in f f }
      \hat{\mono{2}} \inhabits \mono{int}
  \end{align*}
  Show that the expressions \(\hat{\mono{1}} \mono{ } \hat{\mono{2}}\) and
  \(\lambda \mono{f.} (\mono{f f})\) are ill-typed within \(\Gamma_0\). Could
  these expressions be well-typed in a more powerful type system?
\end{problem}

\begin{solution}
  Let \(\mathbb{Z} = \mono{int}\).
  \ \\

  \noindent
  \(\Gamma_0 \entails \lambda \mono{z.z} \inhabits \mathbb{Z} \to \mathbb{Z}\)
  is valid in the simply-typed lambda calculus, as the following derivation does
  not make use of (DM-Gen) or (DM-Inst).

  Derivation:
  \begin{prooftree}
    \AxiomC{\(\{\Gamma_0; \mono{z} \inhabits \mathbb{Z}\}(\mono{z}) =
      \mathbb{Z}\)}
    \RightLabel{\small (DM-Var)}
    \UnaryInfC{\(\Gamma_0; \mono{z} \inhabits \mathbb{Z} \entails \mono{z}
      \inhabits \mathbb{Z}\)}
    \RightLabel{\small (DM-Abs)}
    \UnaryInfC{\(\Gamma_0 \entails \lambda \mono{z.z} \inhabits \mathbb{Z} \to
      \mathbb{Z}\)}
  \end{prooftree}
  \ \\

  \noindent
  \(\Gamma_0 \entails \lambda \mono{z.z} \inhabits \forall \mono{X.X} \to
  \mono{X}\)
  is not valid in the simply-typed lambda calculus because the STLC lacks type
  schemes.

  Derivation:
  \begin{prooftree}
    \AxiomC{\(\{\Gamma_0; \mono{z} \inhabits \mono{X}\}(\mono{z})\)}
    \RightLabel{\small (DM-Var)}
    \UnaryInfC{\(\Gamma_0; \mono{z} \inhabits \mono{X} \entails \mono{z}
      \inhabits \mono{X}\)}
    \RightLabel{\small (DM-Abs)}
    \UnaryInfC{\(\Gamma_0 \entails \lambda \mono{z.z} \inhabits \mono{X} \to
      \mono{X}\)}
    \AxiomC{\(\mono{X} \disjoint ftv(\Gamma_0)\)}
    \RightLabel{\small (DM-Gen)}
    \BinaryInfC{\(\Gamma_0 \entails \lambda \mono{z.z} \inhabits \forall
      \mono{X.X} \to \mono{X}\)}
  \end{prooftree}
  \ \\

  \noindent
  \(\Gamma_0 \entails \mono{let f} = \lambda \mono{z.z }
    \hat{\mono{+}} \mono{ } \hat{\mono{1}} \mono{ in f } \hat{\mono{2}}
    \inhabits \mono{int}\)
  is valid in the simply-typed lambda calculus, as the following derivation does
  not make use of (DM-Gen) or (DM-Inst).

  Final derivation:
  \begin{prooftree}
    \AxiomC{\(\Gamma_0 \entails \lambda \mono{z.z } \hat{\mono{+}} \mono{ }
      \hat{\mono{1}} \inhabits \mathbb{Z} \to \mathbb{Z}\)}
    \AxiomC{\(\Gamma_0; \mono{f} \inhabits \mathbb{Z} \to \mathbb{Z} \entails
      \mono{f } \hat{\mono{2}} \inhabits \mathbb{Z}\)}
    \RightLabel{\small (DM-Let)}
    \BinaryInfC{\(\Gamma_0 \entails \mono{let f} = \lambda \mono{z.z }
      \hat{\mono{+}} \mono{ } \hat{\mono{1}} \mono{ in f } \hat{\mono{2}}
      \inhabits \mathbb{Z}\)}
  \end{prooftree}

  Derivation of first hypothesis:
  \begin{prooftree}
    \AxiomC{\(\{\Gamma_0; \mono{z} \inhabits \mathbb{Z}\}(\hat{\mono{+}}) =
      \mathbb{Z} \to \mathbb{Z} \to \mathbb{Z}\)}
    \RightLabel{\small (DM-Var)}
    \UnaryInfC{\(\Gamma_0; \mono{z} \inhabits \mathbb{Z} \entails \hat{\mono{+}}
      \inhabits \mathbb{Z} \to \mathbb{Z} \to \mathbb{Z}\)}
    \AxiomC{\(\{\Gamma_0; \mono{z} \inhabits \mathbb{Z}\}(\mono{z}) =
      \mathbb{Z}\)}
    \RightLabel{\small (DM-Var)}
    \UnaryInfC{\(\Gamma_0; \mono{z} \inhabits \mathbb{Z} \entails \mono{z}
      \inhabits \mathbb{Z}\)}
    \RightLabel{\small (DM-App)}
    \BinaryInfC{\(\Gamma_0; \mono{z} \inhabits \mathbb{Z} \entails
      \hat{\mono{+}} \mono{ z} \inhabits \mathbb{Z} \to \mathbb{Z}\)}
    \AxiomC{\(\{\Gamma_0; \hat{\mono{z}} \inhabits \mathbb{Z}\}(\mono{1}) =
      \mathbb{Z}\)}
    \RightLabel{\small (DM-Var)}
    \UnaryInfC{\(\Gamma_0; \hat{\mono{z}} \inhabits \mathbb{Z} \entails
      \hat{\mono{1}} \inhabits \mathbb{Z}\)}
    \RightLabel{\small (DM-App)}
    \BinaryInfC{\(\Gamma_0; \mono{z} \inhabits \mathbb{Z} \entails \mono{z }
      \hat{\mono{+}} \mono{ } \hat{\mono{1}} \inhabits \mathbb{Z}\)}
    \RightLabel{\small (DM-Abs)}
    \UnaryInfC{\(\Gamma_0 \entails \lambda \mono{z.z } \hat{\mono{+}} \mono{ }
      \hat{\mono{1}} \inhabits \mathbb{Z} \to \mathbb{Z}\)}
  \end{prooftree}

  Derivation of second hypothesis:
  \begin{prooftree}
    \AxiomC{\(\{\Gamma_0; \mono{f} \inhabits \mathbb{Z} \to \mathbb{Z}\}(
      \mono{f}) = \mathbb{Z} \to \mathbb{Z}\)}
    \RightLabel{\small (DM-Var)}
    \UnaryInfC{\(\Gamma_0; \mono{f} \inhabits \mathbb{Z} \to \mathbb{Z} \entails
      \mono{f} \inhabits \mathbb{Z} \to \mathbb{Z}\)}
    \AxiomC{\(\{\Gamma_0; \mono{f} \inhabits \mathbb{Z} \to \mathbb{Z}\}(
      \hat{\mono{2}}) = \mathbb{Z}\)}
    \RightLabel{\small (DM-Var)}
    \UnaryInfC{\(\Gamma_0; \mono{f} \inhabits \mathbb{Z} \to \mathbb{Z} \entails
      \hat{\mono{2}} \inhabits \mathbb{Z}\)}
    \RightLabel{\small (DM-App)}
    \BinaryInfC{\(\Gamma_0; \mono{f} \inhabits \mathbb{Z} \to \mathbb{Z}
      \entails \mono{f } \hat{\mono{2}} \inhabits \mathbb{Z}\)}
  \end{prooftree}
  \ \\

  \noindent
  Derivation of \(\Gamma_0 \entails \mono{let f} = \lambda \mono{z.z}
    \mono{ in f f } \hat{\mono{2}} \inhabits \mathbb{Z}\)
  is not valid in the simply-typed lambda calculus because mapping
  \(\mono{let}\) to \(\lambda\) would force the two occurrences of \(\mono{f}\)
  to have the same first-order type.

  Final derivation:
  \begin{prooftree}
    \AxiomC{\(\Gamma_0 \entails \lambda \mono{z.z} \inhabits \forall \mono{X.X}
      \to \mono{X}\)}
    \AxiomC{\(\Gamma_0; \mono{f} \inhabits \forall \mono{X.X} \to \mono{X}
      \entails \mono{f f} \inhabits \mathbb{Z} \to \mathbb{Z}\)}
    \AxiomC{\(\{\Gamma_0; \mono{f} \inhabits \forall \mono{X.X} \to \mono{X}\}(
      \hat{\mono{2}}) = \mathbb{Z}\)}
    \RightLabel{\small (DM-Var)}
    \UnaryInfC{\(\Gamma_0; \mono{f} \inhabits \forall \mono{X.X} \to \mono{X}
      \entails \hat{\mono{2}} \inhabits \mathbb{Z}\)}
    \RightLabel{\small (DM-App)}
    \BinaryInfC{\(\Gamma_0; \mono{f} \inhabits \forall \mono{X.X} \to \mono{X}
      \entails \mono{f f } \hat{\mono{2}} \inhabits \mathbb{Z}\)}
    \RightLabel{\small (DM-Let)}
    \BinaryInfC{\(\Gamma_0 \entails \mono{let f} = \lambda \mono{z.z in f f }
      \hat{\mono{2}} \inhabits \mathbb{Z}\)}
  \end{prooftree}

  Derivation of first hypothesis:
  \begin{prooftree}
    \AxiomC{\(\{\Gamma_0; \mono{z} \inhabits \mono{X}\}(\mono{z}) = \mono{X}\)}
    \RightLabel{\small (DM-Var)}
    \UnaryInfC{\(\Gamma_0; \mono{z} \inhabits \mono{X} \entails \mono{z} \inhabits
      \mono{X}\)}
    \RightLabel{\small (DM-Abs)}
    \UnaryInfC{\(\Gamma_0 \entails \lambda \mono{z.z} \inhabits \mono{X} \to
      \mono{X}\)}
    \AxiomC{\(\{\mono{X}\} \disjoint ftv(\Gamma_0)\)}
    \RightLabel{\small (DM-Gen)}
    \BinaryInfC{\(\Gamma_0 \entails \lambda \mono{z.z} \inhabits \forall
      \mono{X.X} \to \mono{X}\)}
  \end{prooftree}

  Derivation of second hypothesis:
  \begin{prooftree}
    \AxiomC{\(\{\Gamma_0; \mono{f} \inhabits \forall \mono{X.X} \to
      \mono{X}\}(\mono{f}) = \forall \mono{X.X} \to \mono{X}\)}
    \RightLabel{\small (DM-Var)}
    \UnaryInfC{\(\Gamma_0; \mono{f} \inhabits \forall \mono{X.X} \to \mono{X}
      \entails \mono{f} \inhabits \forall \mono{X.X} \to \mono{X}\)}
    \RightLabel{\small (DM-Inst)}
    \UnaryInfC{\(\Gamma_0; \mono{f} \inhabits \forall \mono{X.X} \to \mono{X}
      \entails \mono{f} \inhabits (\mathbb{Z} \to \mathbb{Z}) \to (\mathbb{Z}
      \to \mathbb{Z})\)}
    \AxiomC{\(\{\Gamma_0; \mono{f} \inhabits \forall \mono{X.X} \to
      \mono{X}\}(\mono{f}) = \forall \mono{X.X} \to \mono{X}\)}
    \RightLabel{\small (DM-Var)}
    \UnaryInfC{\(\Gamma_0; \mono{f} \inhabits \forall \mono{X.X} \to \mono{X}
      \entails \mono{f} \inhabits \forall \mono{X.X} \to \mono{X}\)}
    \RightLabel{\small (DM-Inst)}
    \UnaryInfC{\(\Gamma_0; \mono{f} \inhabits \forall \mono{X.X} \to \mono{X}
      \entails \mono{f} \inhabits \mathbb{Z} \to \mathbb{Z}\)}
    \RightLabel{\small (DM-App)}
    \BinaryInfC{\(\Gamma_0; \mono{f} \inhabits \forall \mono{X.X} \to \mono{X}
      \entails \mono{f f} \inhabits \mathbb{Z} \to \mathbb{Z}\)}
  \end{prooftree}
  \ \\

  \noindent
  Finally, the ill-typed terms. \(\hat{\mono{1}} \mono{ } \hat{\mono{2}}\) is
  ill-typed because \(\Gamma_0(\hat{\mono{1}}) = \mathbb{Z}\), but its use as
  \(\mono{t}_1\) in (DM-App) would imply that it must be a function type.
  Similarly, the application of \(\mono{f}\) to itself would imply that both
  \(\mono{f} \inhabits \mono{T}\) and that \(\mono{f} \inhabits \mono{T} \to
  \mono{T}^\prime\), but these types are not equal as first order terms.
\end{solution}

\end{document}
